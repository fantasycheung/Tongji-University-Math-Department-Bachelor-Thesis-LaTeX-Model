\chapter{Black-Scholes方程的假设以及研究对象的分析}

\section{B-S方程的假设}
从73年美国芝加哥大学学者提出B-S 期权定价模型以来,各种新型期权纷纷涌现,但是不管怎样变化都是万变不离其宗,基于基本的B-S方程。当然,B-S 方程并不能完全符合金融市场中的实际情况,但是它的定价理论能够在下述假设下,较真实地反映期权应该具备的价值,并直接应用于金融机构对各种期权的定价。
\begin{tjuhypothsis}
	原生资产价格演化遵循几何Brown运动,即满足如下方程:
	\begin{displaymath}
		dS=\mu Sdt+\sigma SdW
	\end{displaymath}
	$\mu$----期望回报率(常数)\\
	$\sigma$----波动率(常数)\\
	$dW$----标准Brown运动
	\begin{displaymath}
		E(dW)=0
	\end{displaymath}
	$$ Var(dW)=dt $$
\end{tjuhypothsis}
\begin{tjuhypothsis}
	没有交易费用或税收。在现实的金融市场中交易费用和税收都是无法避免的,可是由于许多期权都是大额交易,相对而言,这些费用可以在一定程度上不予考虑。
\end{tjuhypothsis}
\begin{tjuhypothsis}
	市场是公平的、透明的,不存在无风险套利机会。虽然金融市场趋于完善,但是即便在金融业极度发达的国家,无风险套利机会都还是存在的。由于人们都是很理性的,这种套利机会一旦出现立刻被大众所得,根据供求平衡理论可以理解,套利机会会在出现后的短时间内消失。
\end{tjuhypothsis}
\begin{tjuhypothsis}
	原生资产不支付股息。
\end{tjuhypothsis}
\begin{tjuhypothsis}
	无风险利率是常数,即无论到期日为何时,利率均不变。
\end{tjuhypothsis}

\section{研究对象分析}
本文的研究对象为招商银行的股票挂钩型理财产品,该产品与中石油,中移动,中国人寿3支股票挂钩,投资期为2年,认购金额为5万元人民币,每6个月为一个计息期,具有自动终止条款,投资者无权提前终止。这个金融产品是挂钩型理财产品中的保障本金投资产品,并且挂钩3支股票价格,无论票的涨跌,本金都是得到保障的。具体的挂钩股票的条款是:在6个月的观察日,如果3支股票的收盘价格大于或者等于收盘价格的100\%,投资者就可以获得最高的当期理财收益,即:当期理财收益=(理财本金的5\%)/2;如果3支股票的收盘价格大于或者等于收盘价格的95\%,即:当期理财收益=(理财本金的3.5\%)/2;否则,当期收益为0。

