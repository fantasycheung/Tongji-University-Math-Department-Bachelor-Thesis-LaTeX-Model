\chapter{前言}

\section{股票挂钩型理财产品}

股票挂钩投资品,又称连动式投资产品或结构型投资产品,是一种收益与股票价格或股价指数等标的相挂钩的结构化产(Structure Products)。通过财务工程技术,针对投资者对市场之不同预期,以拆解或组合衍生性金融商品如股票、一篮子股票、指数、一篮子指数等,并搭配零息债券的方式组合而成的各种不同报酬型态的金融商品。\cite{optionpricing}

\section{挂钩型理财产品市场现状}

挂钩型理财产品发展历史不长,但发展极为迅速,已实现在美国证券交易所、伦敦证券交易所、香港联交易所等处上市交易,是当今国际金融市场上最有发展潜力的业务之一。目前,此类产品已开始中国境内销售,有望成为优良的理财产品以及权证的背对产品,市场前景可观。\cite{On-Manifold-Regularization}

挂钩股指,如汇丰银行此前发行的新品则挂钩股指的绝对值,只要其对应的恒生中国企业指数波动达到一定的幅度,投资者就能获得相应的回报。

在挂钩范围上,此类产品的选择也越来越多,不断有新的热点涌现。如荷兰银行推出的挂钩房地产股票篮子以及水资源股票的产品、渣打银行挂钩媒体篮子股票的产品以及最近民生银行推出的挂钩香港中国概念金融板块股票的非凡人民币理财产品等,使得挂钩股票型产品的投资方向日渐多样化。有些产品的收益相当不错,如荷兰银行去年6月推出的第一期挂钩房地产股票篮子的结构性产品,截至11月2日,该产品的美元和欧元的收益率已分别达到12\%和8\% 。

与市场上越来越多的短期理财产品相比,挂钩股票型的理财产品投资期限普遍较长,大部分产品的投资期限都是在2年左右。这主要是由产品挂钩股票的特点所决定的,时间太短,一旦股价出现较大的波动,不但不能给投资者带来收益,反而会带来损失。因此一段较长的投资期限是获取收益、平摊风险的必需条件。另一方面,银行在发行挂钩股票型产品时都承诺本金的保证,但前提是投资者必须要持有理财计划到期,否则本金就会有损失。

\section{期权}
\subsection{期权的概念}
期权是20世纪70年代中期在美国出现的一种金融创新工具,30多年来,它作为一种防范风险和投机的有效手段而得到迅猛发展。

期权是一种极为特殊的衍生产品,它能使买方有能力避免坏的结果,而从好的结果中获益,同时.它也能使卖方产生巨大的损失。当然,期权不是免费的,这就产生了期权定价问题。期权定价理论是现代金融理论最为重要的成果之一。\cite{optionpricing}它集中体现了金融理论的许多核心问题,其理论之深,方法之多,应用之广,令人惊叹。期权的标的资产也由股票、指数、期货台约、商品(金属、黄金、石油等),外汇增加到了利率,可转换债券、认股权证、掉期和期权本身等许多可交易证券和不可交易证券.期权是一种企业、银行和投资者等进行风险管理的有力工具。

\subsection{期权和期权理论的历史}
期权的理论与实践并非始于1973年Black-Scholes关于期权定价理论论文的发表。早在公元前1200年的古希腊和古腓尼基国的贸易中就已经出现了期权交易的雏形,只不过当时条件下不可能对其有深刻认识。期权的思想萌芽也可以追溯到公元前1800年的《汉穆拉比法典》。公认的期权定价理论的始祖是法国数学家巴舍利耶 (LouisBachelier,1900年)。令人难以理解的是.长达半个世纪之久巴舍利耶的工作没有引起金融界的重视,直到l956年被克鲁辛格(Kruizenga)再次发现。\cite{kexuebao}

